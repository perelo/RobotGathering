\documentclass[11pt, a4paper]{article}

\usepackage[english,francais]{babel}
\usepackage[utf8]{inputenc}
\usepackage[T1]{fontenc}
\usepackage[pdftex]{graphicx}
\usepackage{setspace}
\usepackage[french]{varioref}
\usepackage{amsfonts}
\usepackage{amssymb}
\usepackage{geometry}
\geometry{margin=2cm}

\usepackage{tikz}
\usetikzlibrary{calc}   % coordinate calculation
\usepackage{xifthen}

\title{Rassemblement d'agents mobiles}
\author{\'Eloi Perdereau}
%\date{}

\newcommand{\case}[3] {
  \begin{tikzpicture}[thick, scale=0.6]
    \def \step {1}
    \def \cc {\step/2}  % center of cell
    \coordinate (offset) at ($#1 + (\cc,\cc)$);
    \draw[step=\step] #1 grid ($#1 + (3,3)$);   % draw the grid, base at #1
    % draw the center circle
    \draw ($(1,1) + (offset)$) circle (\cc*0.8);
    % draw the neighbors
    \foreach \coord in #2 {
      \coordinate[at=\coord, name=A];
      \draw ($(A) + (offset)$) circle ({\cc*0.8});
    }
    % draw the movement arrow if #3 is not empty
    \ifthenelse{\equal{#3}{}}{}{
    \draw[->] ($(1,1) + (offset)$) -- ($#3 + (offset)$);
    }
  \end{tikzpicture}
}

\begin{document}

\maketitle

\section{Cas}

Les cas symétriques sont omis.

\subsection{1 voisins}
\case {(0,0)}  {{(0,2)}}                   {(0,2)}
\case {(4,0)}  {{(1,2)}}                   {(1,2)}

\subsection{2 voisins}
\case {( 0,0)} {{(0,2),(1,2)}}             {(1,2)}
\case {( 4,0)} {{(0,1),(1,2)}}             {(0,2)}
\case {( 8,0)} {{(0,2),(2,2)}}             {(1,2)}
\case {(12,0)} {{(0,2),(2,1)}}             {}
\case {(16,0)} {{(0,2),(2,0)}}             {}
\case {(20,0)} {{(1,2),(1,0)}}             {}

\subsection{3 voisins}
\case {( 0,0)} {{(0,2),(1,2),(2,2)}}       {(1,2)}
\case {(20,0)} {{(0,2),(1,2),(0,1)}}       {(0,2)}
\case {( 4,0)} {{(0,2),(1,2),(2,1)}}       {(1,2)}
\case {(16,0)} {{(0,2),(1,2),(0,0)}}       {}
\case {(12,0)} {{(0,2),(1,2),(1,0)}}       {}
\case {( 8,0)} {{(0,2),(1,2),(2,0)}}       {} \\

\case {( 0,4)} {{(0,2),(2,2),(2,0)}}       {}
\case {( 4,4)} {{(0,2),(2,2),(1,0)}}       {}
\case {( 8,4)} {{(1,2),(0,1),(2,1)}}       {}
\case {(12,4)} {{(1,2),(0,1),(2,0)}}       {}

\subsection{4 voisins}
\case {(0,0)} {{(0,2),(1,2),(2,2),(2,1)}} {(1,2)}
\case {(0,0)} {{(0,2),(1,2),(2,2),(2,0)}} {}
\case {(0,0)} {{(0,2),(1,2),(2,2),(1,0)}} {} \\

\case {(0,0)} {{(0,2),(1,2),(2,1),(2,0)}} {}
\case {(0,0)} {{(0,2),(1,2),(2,1),(1,0)}} {}
\case {(0,0)} {{(0,2),(1,2),(2,1),(0,0)}} {}
\case {(0,0)} {{(0,2),(1,2),(2,1),(0,1)}} {} \\

\case {(0,0)} {{(0,2),(1,2),(2,0),(1,0)}} {}
\case {(0,0)} {{(0,2),(1,2),(2,0),(0,0)}} {}
\case {(0,0)} {{(0,2),(1,2),(2,0),(0,1)}} {}
\case {(0,0)} {{(0,2),(1,2),(1,0),(0,0)}} {}
\case {(0,0)} {{(0,2),(2,2),(2,0),(0,0)}} {}
\case {(0,0)} {{(1,2),(2,1),(1,0),(0,1)}} {}


\subsection{5 voisins et plus}
Pour énumérer les cas avec 5 et 6 robots voisins, il suffit de prendre le
complémentaire des cas avec respectivement 3 et 2 robots voisins.
Aucun de ces cas n'entraine un mouvement de la part du robot concerné.


\begin{figure}[h]
\centering

\end{figure}


\end{document}
