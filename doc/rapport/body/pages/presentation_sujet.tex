\newcommand{\Observer}{\textit{Observer}\xspace}
\newcommand{\Calculer}{\textit{Calculer}\xspace}
\newcommand{\SeDeplacer}{\textit{Se\_déplacer}\xspace}

\newcommand{\outil}[1]{\textit{#1}\xspace}

\section{Cadre du projet}

\subsection{Présentation du sujet}

Bien que le problème ne soit théorique, nous modélisons un système plus ou
moins réaliste qui pourrait émerger d'applications pratiques. Néanmoins, nous
allons nous concentrer sur le développement d'un algorithme correct pour
résoudre le problème posé sans se soucier des cas pratiques.

Comme énoncé dans l'introduction, nous nous plaçons dans le cadre d'un univers
discret. C'est à dire que le plan peut être représenté par une grille infinie à
deux dimension où chaque cellule possède 8 cellules adjacentes différentiables
par leur coordonnées relatives. Par convention, on définit l'axe $x$
\textit{vers la droite} et l'axe $y$ \textit{vers le bas}. Les entités mobiles
(ou robots) sont modélisés comme des unités de calcul ayant une mémoire locale
et sont capables d'effectuer des calculs locaux. Les robots sont placés dans le
plan et sont représentés par des cellules "pleines" ayant un couple de
coordonnées $(x,y)$ dans $\mathbb{N}^2$. Ils sont capables de se déplacer
librement dans l'espace de manière synchrone et sont dotés de capacités
sensorielles leur permettant d'observer la position de leur voisin à un instant
$t$. Plus formellement, les robots effectuent en permanence une succession de
trois opérations~:
\begin{enumerate}[(i)]
  \item \Observer. Le robot observe son voisinage en activant ses capteurs qui
  lui renvoient un instantané des positions des robots à l'intérieur de son
  champ de visibilité.
  \item \Calculer. Le robot effectue un calcul local donné par l'algorithme. Le
  résultat de ce calcul est un point de destination (à distance au plus 1 du
  robot concerné). Si ce point est la position du robot, celui-ci ne se déplace
  pas.
  \item \SeDeplacer. Le robot se déplace à la position renvoyée par le calcul.
  \\
\end{enumerate}

La séquence \Observer-\Calculer-\SeDeplacer constitue un \textit{cycle de
calcul}, aussi appelé \textit{ronde} ou \textit{étape}. On définit plusieurs
contraintes et suppositions sur les robots dont il faut nous soumettre.

\paragraph{Synchronicité} Le temps est divisé en étapes (discrétisation). \`A
chaque étape de temps, les robots effectuent une ronde, i.e. les trois
opérations cités ci-dessus.

\paragraph{Pas de communication directe entre les robots} Certains modèles
permettent l'envoi de messages entre les robots en plus de leur déplacement.
Cela est en effet assez réaliste si les robots sont considérés comme des
machines électroniques avec des capacités de communications et un protocole
défini. Ils communiquent néanmoins indirectement via leur mouvements et leur
positions relatives.

\paragraph{Visibilité limitée} Les robots ne sont capables de recevoir de leur
capteurs uniquement des informations sur leur entourage restreint. Ils n'ont
donc pas de connaissance du nombre total de robots où de la présence de robots
en dehors de leur champ de visibilité (\textit{visibility radius}.)

\paragraph{Homogénéité} Tous les robots sont pré-programmés avec un même
algorithme et commencent à la même étape. Le système étant synchrone, ils
peuvent garder localement un compteur de cycle qu'ils effectuent, et ce
compteur aura la même valeur pour tous les robots à chaque instant de
l'algorithme.

\paragraph{Anonymat} Il n'existe aucun paramètre global permettant de dissocier
les robots. Leur capteurs leur renvoie donc que des informations sur la
position des robots alentours, et nullement des particularités propres à chaque
robot voisin.

\paragraph{Déterminisme} Les voisinages déterminent le mouvement~: pour un même
voisinage, les robots réagissent de la même façon. L'algorithme développé sera
donc déterministe. Une solution au \GatheringProblem a déjà été apporté dans
un contexte non-déterministe.

\paragraph{Points denses} Plusieurs robots peuvent se retrouver dans une même
position a un instant $t$. Il n'est pas nécessaires pour un robot de connaitre
le nombre de robots sur un point particulier, car une fois qu'ils sont à la
même position, tous recevrons les mêmes informations de leur capteurs, et
prendrons donc la même décision du fait de leur homogénéité. On peut donc les
considérer comme un seul robot.

\paragraph{Mémoire constante} Chaque robot ne peut retenir en mémoire qu'un
nombre limité d'informations. Par exemple, son dernier mouvement ou son
entourage précédent.

\paragraph{Déplacement local} \`A chaque étape, les robots ne peuvent se
déplacer que sur une cellule adjacente.

\paragraph{Désorientation} En plus d'avoir un système de coordonnée différente,
les robots ont également leur propre orientation. Nous ne pouvons donc pas nous
fier à l'orientation des instantanés reçues sur les entourages et devons les
considérer comme s'ils avaient subies aléatoirement des rotations successives
des $90^{\circ}$.

\paragraph{Autonomie} Le mécanisme de coordination utilisé par les robots pour
se rassembler doit être totalement décentralisé, i.e. aucun contrôle central
n'est utilisé.

\paragraph{Terminaison} Du fait de l'autonomie des robots~: à la fin de chaque
étape, chaque robot doit savoir s'il est dans un état terminal ou non. Il n'y a
donc pas de système global permettant la désactivation des robots à distance.
Quand tous les robots sont dans un état terminal, ils s'arrêtent et
l'algorithme se termine. Quand aucun robot ne se déplace lors d'une étape, le
système n'évoluera plus~; on dit qu'on a atteint un état \textit{quescient}. \\

Le graphe défini comme sous graphe de l'espace ne contenant que les n\oe{}uds
occupés par des robots est connexe au départ, et doit le rester après chaque
cycle. C'est à dire que chaque robot est accessible par tous les autres en ne
passant que par des robots à distance 1. Cette restriction est nécessaire car,
intuitivement, il est très difficile, voire impossible pour un robot qui n'a
pas de voisin de se rassembler avec les autres robots de l'espace dû à sa
visibilité limitée. On peut dire qu'on a un point de rassemblement par
composante connexe.

Concernant ce point de rassemblement, on ne peut pas garantir son unicité à
cause de la désorientation des robots. Par exemple, s'il ne reste que deux
robots adjacents, ils ont le même voisinage (à une rotation près), et donc à
cause de leur déterminisme, il n'y a pas moyen de les dissocier pour choisir un
point unique pour le rassemblement. Nous considérons donc que s'il y a 1, 2 ou
4 robots adjacents, l'algorithme peut se terminer et les robots sont
rassemblés. Plus précisément, les cas de terminaison sont exposés en figure
\ref{fig:end_cases}. \\ \begin{figure}[h!]
  \centering
  \begin{subfigure}[b]{0.16\textwidth}
    \resizebox{\linewidth}{!} {
      \spacee {(3,3)} {{(1,1)}} {{}}
    }
  \end{subfigure}
  \begin{subfigure}[b]{0.20\textwidth}
    \resizebox{\linewidth}{!} {
      \spacee {(4,3)} {{(1,1),(2,1)}} {{}}
    }
  \end{subfigure}
  \begin{subfigure}[b]{0.20\textwidth}
    \resizebox{\linewidth}{!} {
      \spacee {(4,4)} {{(1,1),(2,2)}} {{}}
    }
  \end{subfigure}
  \begin{subfigure}[b]{0.20\textwidth}
    \resizebox{\linewidth}{!} {
      \spacee {(4,4)} {{(1,1),(2,1),(1,2),(2,2)}} {{}}
    }
  \end{subfigure}
\caption{Cas terminaux globaux}
\label{fig:end_cases}
\end{figure}


Pour résumer, on a un ensemble de robots sur le plan discret disposé de manière
connexe qui doivent se rassembler. Ils se déplacent de manière synchrone, sont
autonomes, homogènes, anonymes et désorientés. L'objectif du sujet est de
trouver un algorithme déterministe à exécuter par tous les robots, qui parvient
à les rassembler en un nombre de rondes fini. Je me suis d'abord attaché à
déterminer les cas de voisinage à distance 1, tout en implémentant l'interface
graphique permettant une visualisation et la mise en \oe{}uvre de l'algorithme.
Puis il a fallu modifier, sophistiquer et prouver la correction de
l'algorithme.

\subsection{Déroulement du projet}

Le TER a commencé le 10 mars, étant réalisé seul, et la programmation assez
légère, je n'ai pas eu à utiliser de méthodes de développement très
développées. J'ai néanmoins utilisé un système de gestion de version tout au
long du projet. Cela m'a permis d'avoir les codes sources et toute l'historique
de mon travail où que je sois (car hébergé sur internet). De plus, cela rendais
la communication avec mon tuteur plus simple, il pouvais voir le travail que je
réalisais à tout moment.

Une fois la base de code écrite, elle faisait l'objet de beaucoup de petites
modifications temporaires pour tester les avancement de recherche effectués à
côté. La preuve de l'algorithme a également fait appel à la programmation, mais
cela était toujours temporaire et n'affectais pas l'application graphique, d'où
la nécessité de versionner les sources.

Enfin, la soutenance s'est tenue le 2 juin devant un jury composé de membres du
laboratoire d'informatique fondamentale (LIF).

\subsection{Outils}

Je vais vous présenter ici les outils utilisés durant le projet. La majorité
sont des logiciels libres, car je pense que les privilégier est un devoir moral
de tout utilisateur.

\paragraph{Environnement de développement} \outil{Vim} et \outil{tmux} dans un
système \outil{GNU/Linux}. Vim (Vi iMproved) est un éditeur de texte modal très
personnalisable en mode texte. Il est disponible par défaut sur beaucoup de
distributions GNU/Linux. Tmux est un multiplexeur de terminaux en mode texte
qui permet de manipuler et d'utiliser plusieurs terminaux virtuels dans un seul
\textit{terminal} en tant que processus système.

\paragraph{Rapport} \outil{\LaTeX}. L'édition en mode texte simple (plaintext)
du rapport me permet de le versionner et d'utiliser mon environnement de
développement habituel. Je peux également générer et intégrer facilement au
rapport des schémas à partir du code Python en utilisant le package
\textit{tikz}.

\paragraph{Code} \outil{Python} et la bibliothèque \outil{TkInter}. Python est
un langage interprété et interactif de haut niveau doté d'une grande variété de
modules intégrés par défaut. Parmi ceux-ci, Tkinter~: rapide à prendre en main,
il permet la création d'interfaces graphiques simples.

\paragraph{Versionning} \outil{Git}~: logiciel de gestion de versions
décentralisé permettant une gestion très fine du code source ou de n'importe
quel contenu texte. Le dépôt du projet est hébergé sur \textit{GitHub}
(\url{http://github.com/perelo/RobotGathering}), ainsi que la configuration de
mon environnement de développement. \\

Tous ces outils me permettent de travailler confortablement sur des machines
minimalement équipées, où même à travers une session ssh distante.
