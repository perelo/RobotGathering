\newcommand{\Observer}{\textit{Observer}\xspace}
\newcommand{\Calculer}{\textit{Calculer}\xspace}
\newcommand{\SeDeplacer}{\textit{Se\_déplacer}\xspace}

\section{Cadre du projet}

\subsection{Présentation du sujet}

Bien que le problème ne soit théorique, nous modélisons un système plus ou
moins réaliste qui pourrait émerger d'applications pratiques. Néanmoins, nous
allons nous concentrer sur le développement d'un algorithme correct pour
résoudre le problème posé sans se soucier des cas pratiques.

Comme énoncé dans l'introduction, nous nous plaçons dans le cadre d'un univers
discret. C'est à dire que le plan peut être représenté par un graphe simple
non-orienté où chaque sommet possède 8 voisins différentiables par leur
coordonnées relatives au sommet considéré. Les entités mobiles (ou robots) sont
modélisés comme des unités de calcul ayant une mémoire locale et sont capables
d'effectuer des calculs locaux. Les robots sont placés dans le plan et sont
représentés par des points de $\mathbb{N}^2$. Ils sont capables de se déplacer
librement dans l'espace de manière synchrone et sont dotés de capacités
sensorielles leur permettant d'observer la position de leur voisin à un instant
$t$. Plus formellement, les robots effectuent en permanence une succession de
trois opérations~:
\begin{enumerate}[(i)]
  \item \Observer. Le robot observe son voisinage en activant ses
  capteurs qui lui renvoient un instantané des positions des robots à
  l'intérieur de son champ de visibilité.
  \item \Calculer. Le robot effectue un calcul local donné par
  l'algorithme. Le résultat de ce calcul est un point de destination (à
  distance au plus 1 du robot concerné). Si ce point est la position du robot,
  celui-ci ne se déplace pas.
  \item \SeDeplacer. Le robot se déplace à la position renvoyée par le
  calcul. \\
\end{enumerate}
La séquence \Observer-\Calculer-\SeDeplacer constitue un cycle de calcul. On
définit plusieurs contraintes et suppositions sur les robots dont il faut nous
soumettre.

\paragraph{Synchronicité} Le temps passé pour \Observer, \Calculer et
\SeDeplacer est le même pour tous les robots. De plus, ils commencent à
\Observer en même temps.

\paragraph{Pas de communication directe entre les robots} Certains modèles
permettent l'envoi de messages entre les robots en plus de leur déplacement.
Cela est en effet assez réaliste si les robots sont considérés sont des
machines électroniques avec des capacités de communications et un protocole
défini. Néanmoins, nous n'autorisons ici qu'une communication indirecte avec
leur positions relatives.

\paragraph{Visibilité limitée} Les robots ne sont capables de recevoir de leur
capteurs uniquement des informations sur leur entourage restreint. Ils n'ont
donc pas de connaissance du nombre total de robots où de la présence de robots
en dehors de leur champ de visibilité (\textit{visibility radius}.)

\paragraph{Homogénéité} Tous les robots sont pré-programmés avec un même
algorithme et commencent à la même étape. Le système étant synchrone, ils
peuvent garder localement un compteur de cycle qu'ils effectuent, et ce
compteur aura la même valeur pour tous les robots à chaque instant de
l'algorithme.

\paragraph{Anonymat} Il n'existe aucun paramètre global permettant de dissocier
les robots. Leur capteurs leur renvoie donc que des informations sur la
position des robots alentours, et nullement des particularités propres à chaque
robot voisin.

\paragraph{Points denses} Plusieurs robots peuvent se retrouver dans une même
position a un instant $t$. Il n'est pas nécessaires pour un robot de connaitre
le nombre de robots sur un point particulier, car une fois qu'ils sont à la
même position, tous recevrons les mêmes informations de leur capteurs, et
prendrons donc la même décision du fait de leur homogénéité. On peut donc les
considérer comme un seul robot.

\paragraph{Mémoire constante} Chaque robot ne peut retenir en mémoire qu'un
nombre limité d'informations. Par exemple, son dernier mouvement ou son
entourage précédent.

\paragraph{Déplacement 1} \`A chaque étape, les robots ne peuvent se déplacer que
sur un n\oe{}ud voisin direct.

\paragraph{Désorientation} En plus d'avoir un système de coordonnée différente,
les robots ont également leur propre orientation. Nous ne pouvons donc pas nous
fier à l'orientation des instantanés reçues sur les entourages et devons les
considérer comme s'ils avaient subies aléatoirement des rotations successives
des $90^{\circ}$.

\paragraph{Autonomie} Le mécanisme de coordination utilisé par les robots pour
se rassembler doit être totalement décentralisé, i.e aucun contrôle central
n'est utilisé.

\paragraph{Terminaison} Du fait de l'autonomie des robots~: à la fin de chaque
étape, chaque robot doit savoir s'il est dans un état terminal ou non. Il n'y a
donc pas de système global permettant la désactivation des robots à distance.
Quand tous les robots sont dans un état terminal, ils s'arrêtent et
l'algorithme se termine. Quand aucun robot ne se déplace lors d'une étape, le
système n'évoluera plus~; on dit qu'on a atteint un état quescient. \\

Le graphe défini comme sous graphe de l'espace ne contenant que les n\oe{}uds
occupés par des robots est connexe au départ, et doit le rester après chaque
cycle. C'est à dire que chaque robot est accessible par tous les autres en ne
passant que par des robots à distance 1. Cette restriction est nécessaire car,
intuitivement, il est très difficile, voire impossible pour un robot qui n'a
pas de voisin de se rassembler avec les autres robots de l'espace dû à sa
visibilité limitée. On peut dire qu'on a un point de rassemblement par
composante connexe.

Concernant ce point de rassemblement, on ne peut pas garantir son unicité à
cause de la désorientation des robots. Par exemple, s'il ne reste que deux
robots adjacents, ils ont le même voisinage (à une rotation près), et donc il
n'y a pas moyen de les dissocier pour choisir un point unique pour le
rassemblement. Nous considérons donc que s'il y a 1, 2 ou 4 robots adjacents,
l'algorithme peut se terminer et les robots sont rassemblés. Plus précisément,
les cas de terminaison sont exposés en figure \ref{fig:end_cases}.
\begin{figure}[h!]
  \centering
  \begin{subfigure}[b]{0.16\textwidth}
    \resizebox{\linewidth}{!} {
      \spacee {(3,3)} {{(1,1)}} {{}}
    }
  \end{subfigure}
  \begin{subfigure}[b]{0.20\textwidth}
    \resizebox{\linewidth}{!} {
      \spacee {(4,3)} {{(1,1),(2,1)}} {{}}
    }
  \end{subfigure}
  \begin{subfigure}[b]{0.20\textwidth}
    \resizebox{\linewidth}{!} {
      \spacee {(4,4)} {{(1,1),(2,2)}} {{}}
    }
  \end{subfigure}
  \begin{subfigure}[b]{0.20\textwidth}
    \resizebox{\linewidth}{!} {
      \spacee {(4,4)} {{(1,1),(2,1),(1,2),(2,2)}} {{}}
    }
  \end{subfigure}
\caption{Cas terminaux globaux}
\label{fig:end_cases}
\end{figure}




\subsection{Déroulement du projet}
TODO \\
Le projet est fait seul, et j'utilise un système de gestion de version (Git)
hébergé par Github. Cela me permet d'avoir les sources et toute l'historique de
mon travail où que je sois, et cela rend la communication avec mon tuteur plus
simple, il peut voir le travail que je réalise à tout moment.
\LaTeX -> plaintext -> suivi des modifications des documents.
python tkinter \\
