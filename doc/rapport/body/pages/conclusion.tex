\section*{Conclusion}
\addcontentsline{toc}{section}{Conclusion}

L'objectif du TER qui était de développer une interface graphique et d'étudier
les cas d'instantanés permettant le rassemblement de robot a été atteint très
rapidement. Il s'en est suivi une longue étude pour prouver que le travail
théorique qui a été fait est correct dans tous les cas. \'Etude qui est
d'ailleurs toujours en cours.

J'ai conçu mon programme de la manière la plus claire et simple possible et permettant
une extensibilité maximale. Il pourra donc être réutilisé, modifié et amélioré
pour la visualisation, le test et l'implémentation de multiples autres
algorithmes distribués ayant un contexte d'exécution plus ou moins différent.
En effet, je me suis beaucoup concentré sur l'aspect discret de l'espace et du
temps, mais le code n'étant pas très conséquent, il pourra être facilement
adapté à d'autres contraintes.

De plus, j'ai beaucoup appris sur l'algorithmique distribuée et plus
précisément sur les agents mobiles et le \emph{pattern formation problem} en me
renseignant sur des articles de recherche. Ce fut très encourageant de
travailler sur un sujet si récent (certains articles datent de 2012.)

D'un point de vue personnel, ce projet m'a permis d'encore mieux me
familiariser avec le monde de la recherche (j'avais déjà effectué un
stage au sein du laboratoire en licence.) Enfin, la perspective de publier un
article était très motivante et cela me conforte dans mon choix de faire une
thèse après mon Master. J'ai en effet découvert l'immensité des problèmes de
recherche encore ouvert dans ce domaine. Rien que sur le problème de formation
de motifs par des robots, qui est assez spécifique, il y a beaucoup de
chose à étudier si l'on change les spécificités des robots.

Même s'il me reste
encore beaucoup à apprendre, cela me donne envi d'y participer activement et
d'essayer de faire avancer les choses à mon
échelle\footnote{\url{http://matt.might.net/articles/phd-school-in-pictures/}}.

