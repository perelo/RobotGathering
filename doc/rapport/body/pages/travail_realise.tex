\newcommand{\Gathered}{\ensuremath{\textsc{gathered}}\xspace}

\theoremstyle{plain}
\newtheorem{thm}{Theorem}
\newtheorem{lem}[thm]{Lemma}
\newtheorem{prop}[thm]{Proposition}
\newtheorem*{cor}{Corollary}

\theoremstyle{definition}
\newtheorem{defn}{Definition}[section]
\newtheorem{conj}{Conjecture}[section]
\newtheorem{exmp}{Example}[section]

\theoremstyle{remark}
\newtheorem*{rem}{Remark}
\newtheorem*{note}{Note}

\section{Travail réalisé}

Cette section retrace le travail que j'ai réalisé durant ce projet, les
difficultés que j'ai rencontré ainsi que les solutions apportées.

\subsection{Développement et preuve de l'algorithme}

Le processus de recherche a nécessité plusieurs étapes, et est encore en cours.
Dans un premier temps, il a fallu déterminer les cas de voisinage (instantanés)
dans lesquels les robots effectuent un mouvement. On considère d'abord que les
robots n'ont qu'une visibilité de 1, c'est à dire qu'ils ne voient que les
robots qui sont dans des cellules adjacentes. La figure \ref{fig:cas} en annexe
montre tous les cas de voisinage à distance 1 qu'un robot peut rencontrer. Il
n'est pas nécessaire de traiter les cas analogues correspondant à des rotations
de $90^\circ{}$, $180^\circ{}$ et $270^\circ{}$ de chacun de ces cas.  En
effet, lorsqu'un robot reçoit un instantané qui ne fait pas parti des cas
exposés, il effectue la rotation nécessaire, et applique l'algorithme selon le
cas ayant subi la rotation. \\

L'idée de base pour le rassemblement est d'\textit{arrondir les angles}. La
figure \ref{fig:cas_mvt} montre les cas sujet à mouvements. Les autres cas
de voisinage n'impliquent aucun mouvement de la part des robots.

\begin{figure}
  \centering
  % \captionsetup[subfigure]{labelformat=empty}
  \begin{subfigure}[b]{0.15\textwidth}
    \resizebox{\linewidth}{!} {
      \spacee {(3,3)} {{(1,1),(2,0)}}                   {{{(1,1)/(2,0)}}}
    }
    \caption{}
  \end{subfigure}
  \begin{subfigure}[b]{0.15\textwidth}
    \resizebox{\linewidth}{!} {
      \spacee {(3,3)} {{(1,1),(1,0)}}                   {{{(1,1)/(1,0)}}}
    }
    \caption{}
  \end{subfigure}
  \begin{subfigure}[b]{0.15\textwidth}
    \resizebox{\linewidth}{!} {
      \spacee {(3,3)} {{(1,1),(2,0),(1,0)}}             {{{(1,1)/(1,0)}}}
    }
    \caption{}
  \end{subfigure}
  \begin{subfigure}[b]{0.15\textwidth}
    \resizebox{\linewidth}{!} {
      \spacee {(3,3)} {{(1,1),(0,0),(2,0)}}             {{{(1,1)/(1,0)}}}
    }
    \caption{}
  \end{subfigure}
  \begin{subfigure}[b]{0.15\textwidth}
    \resizebox{\linewidth}{!} {
      \spacee {(3,3)} {{(1,1),(0,0),(1,0)}}             {{{(1,1)/(1,0)}}}
    }
    \caption{}
  \end{subfigure}
  \begin{subfigure}[b]{0.15\textwidth}
    \resizebox{\linewidth}{!} {
      \spacee {(3,3)} {{(1,1),(0,0),(2,1)}}             {{{(1,1)/(1,0)}}}
    }
    \caption{}
  \end{subfigure}
  \begin{subfigure}[b]{0.15\textwidth}
    \resizebox{\linewidth}{!} {
      \spacee {(3,3)} {{(1,1),(2,0),(0,1)}}             {{{(1,1)/(1,0)}}}
    }
    \caption{}
  \end{subfigure}
  \begin{subfigure}[b]{0.15\textwidth}
    \resizebox{\linewidth}{!} {
      \spacee {(3,3)} {{(1,1),(0,1),(1,0)}}             {{{(1,1)/(0,0)}}}
    }
    \caption{}
  \end{subfigure}
  \begin{subfigure}[b]{0.15\textwidth}
    \resizebox{\linewidth}{!} {
      \spacee {(3,3)} {{(1,1),(0,0),(1,0),(2,0)}}       {{{(1,1)/(1,0)}}}
    }
    \caption{}
  \end{subfigure}
  \begin{subfigure}[b]{0.15\textwidth}
    \resizebox{\linewidth}{!} {
      \spacee {(3,3)} {{(1,1),(2,0),(1,0),(2,1)}}       {{{(1,1)/(1,0)}}}
    }
    \caption{}
  \end{subfigure}
  \begin{subfigure}[b]{0.15\textwidth}
    \resizebox{\linewidth}{!} {
      \spacee {(3,3)} {{(1,1),(0,0),(1,0),(2,1)}}       {{{(1,1)/(1,0)}}}
    }
    \caption{}
  \end{subfigure}
  \begin{subfigure}[b]{0.15\textwidth}
    \resizebox{\linewidth}{!} {
      \spacee {(3,3)} {{(1,1),(2,0),(1,0),(0,1)}}       {{{(1,1)/(1,0)}}}
    }
    \caption{}
  \end{subfigure}
  \begin{subfigure}[b]{0.15\textwidth}
    \resizebox{\linewidth}{!} {
      \spacee {(3,3)} {{(1,1),(0,0),(2,1),(2,2)}}       {{{(1,1)/(1,0)}}}
    }
    \caption{}
  \end{subfigure}
  \begin{subfigure}[b]{0.15\textwidth}
    \resizebox{\linewidth}{!} {
      \spacee {(3,3)} {{(1,1),(2,0),(0,1),(0,2)}}       {{{(1,1)/(1,0)}}}
    }
    \caption{}
  \end{subfigure}
  \begin{subfigure}[b]{0.15\textwidth}
    \resizebox{\linewidth}{!} {
      \spacee {(3,3)} {{(1,1),(0,0),(2,1),(2,0)}}       {{{(1,1)/(1,0)}}}
    }
    \caption{}
  \end{subfigure}
  \begin{subfigure}[b]{0.15\textwidth}
    \resizebox{\linewidth}{!} {
      \spacee {(3,3)} {{(1,1),(0,0),(0,1),(2,0)}}       {{{(1,1)/(1,0)}}}
    }
    \caption{}
  \end{subfigure}
  \begin{subfigure}[b]{0.15\textwidth}
    \resizebox{\linewidth}{!} {
      \spacee {(3,3)} {{(1,1),(0,0),(1,0),(2,0),(2,1)}} {{{(1,1)/(1,0)}}}
    }
    \caption{}
  \end{subfigure}
  \begin{subfigure}[b]{0.15\textwidth}
    \resizebox{\linewidth}{!} {
      \spacee {(3,3)} {{(1,1),(0,0),(1,0),(2,0),(0,1)}} {{{(1,1)/(1,0)}}}
    }
    \caption{}
  \end{subfigure}
\caption{Cas de voisinnage sujet à mouvement}
\end{figure}



Ces cas semblent assez naturels, cependant ce n'est pas aussi simple. Par
exemple, examinons de plus près les cas (f) et (g). Si les robots se déplacent
dans ces cas, il existe une infinité d'arrangements des robots qui mènent à un
graphe non connexe. Or, si ils ne se déplacent pas, il existe une infinité
d'arrangements quescient alors que les robots ne sont pas du tout rassemblés.
Il en est de même pour les cas (m) et (n). Les figures \ref{fig:deconnexion} et
\ref{fig:quescient} montrent des exemples de cas symétriques où les robots sont
respectivement déconnectés et quescient si les cas (f) et (g) ne sont
respectivement utilisés et non utilisés. Même s'il n'était pas nécessaire de le
prouver, cela montre que le \GatheringProblem ne pourra se résoudre avec des
robots sans mémoire (\textit{oblivious}.)

\begin{figure}
  \centering
  \begin{subfigure}[b]{0.40\textwidth}
    \resizebox{\linewidth}{!} {
      \spacee {(9,9)} {{(7, 3),(1, 3),(6, 6),(3, 0),(6, 2),(5, 1),(3, 7),(0,
      3),(8, 5),(5, 8),(4, 0),(4, 8),(1, 5),(5, 0),(2, 2),(5, 7),(2, 6),(0,
      4),(7, 5),(0, 5),(8, 3),(3, 1),(3, 8),(8, 4)}} {{{(0, 3)/(1, 4)},{(3,
      1)/(2, 1)},{(5, 7)/(6, 7)},{(3, 0)/(4, 1)},{(3, 8)/(4, 7)},{(0, 5)/(1,
      4)},{(8, 3)/(7, 4)},{(1, 3)/(1, 2)},{(3, 7)/(2, 7)},{(5, 1)/(6, 1)},{(5,
      8)/(4, 7)},{(8, 5)/(7, 4)},{(7, 5)/(7, 6)},{(5, 0)/(4, 1)},{(7, 3)/(7,
      2)},{(1, 5)/(1, 6)}}}
    }
  \end{subfigure}
  \begin{subfigure}[b]{0.40\textwidth}
    \resizebox{\linewidth}{!} {
      \spacee {(9,9)} {{(1, 2),(2, 7),(4, 7),(2, 6),(6, 7),(6, 6),(6, 1),(4,
      8),(1, 4),(7, 4),(6, 2),(1, 6),(2, 1),(2, 2),(0, 4),(8, 4),(4, 1),(7,
      6),(7, 2),(4, 0)}} {{}}
    }
  \end{subfigure}
  \caption{Déconnexion}
\end{figure}


\begin{figure}
  \centering
  \resizebox{0.30\textwidth}{!} {
    \spacee {(7,7)}{{(6, 4),(2, 6),(4, 6),(2, 0),(3, 0),(0, 2),(6, 3),(1, 5),(6,
    2),(3, 6),(0, 4),(5, 1),(0, 3),(4, 0),(1, 1),(5, 5)}} {{}}
  }
\caption{Quescient}
\end{figure}



Pour résoudre ce problème, il faut réussir à détecter la déconnexion, puis
utiliser la mémoire des robots pour revenir à la position précédente. Cela
nécessite deux rondes~:
\begin{enumerate}[1]
  \item Regarder son entourage et se déplacer en utilisant les cas (f), (g),
  (m) ou (n) s'il le faut, et enregistrer le mouvement.
  \item Détecter la déconnexion si on était dans un des cas (f), (g), (m) ou
  (n) et se déplacer à la position précédente si nécessaire.
\end{enumerate}
Il faut que tous les robots exécutent ces deux rondes, et qu'ils ne se
déplacent pas à la deuxième ronde s'ils ne sont pas déconnectés. Cela peut se
réaliser facilement avec un compteur de rondes~: Comme les robots commencent
tous en même temps, on peut définir que les rondes paires correspondent à la
première étape, et les rondes impaires à la deuxième. Ainsi, cela double le
nombre de rondes sans affecter la complexité de l'algorithme.

Si un robot en position ($i$,$j$) est dans le cas (f) ou (m) à la première
étape, la déconnexion est détectée dans la deuxième étape par l'absence totale
de robots aux positions ($i+1$,$j$), ($i+1$,$j-1$) et ($i$,$j-1$). Pour les cas
(g) et (n) ce sont les positions ($i-1$,$j$), ($i-1$,$j-1$) et ($i$,$j-1$). \\

Ceci étant réglé, un autre problème se pose~: il existe des paires
d'arrangements circulaires de robots qui alternent entre eux. Autrement dit,
les robots ne sont pas quescient mais ne se rassemblent jamais. La figure
\ref{fig:oscillation} donne un tel exemple d'oscillation.

\begin{figure}
  \centering
  \begin{subfigure}[b]{0.40\textwidth}
    \resizebox{\linewidth}{!} {
      \spacee {(9,9)} {{(7, 3), (5, 7), (4, 8), (7, 7), (2, 1), (1, 3), (1, 6),
      (5, 1), (3, 7), (7, 2), (4, 0), (1, 2), (6, 7), (7, 6), (1, 5), (0, 4),
      (1, 1), (7, 1), (7, 5), (2, 7), (6, 1), (3, 1), (1, 7), (8, 4)}} {{{(1,
      5)/(0, 5)},{(5, 7)/(5, 8)},{(5, 1)/(5, 0)},{(0, 4)/(1, 4)},{(7, 7)/(6,
      6)},{(7, 1)/(6, 2)},{(3, 1)/(3, 0)},{(1, 1)/(2, 2)},{(1, 3)/(0, 3)},{(1,
      7)/(2, 6)},{(8, 4)/(7, 4)},{(7, 5)/(8, 5)},{(4, 8)/(4, 7)},{(7, 3)/(8,
      3)},{(3, 7)/(3, 8)},{(4, 0)/(4, 1)}}}
    }
  \end{subfigure}
  \begin{subfigure}[b]{0.40\textwidth}
    \resizebox{\linewidth}{!} {
      \spacee {(9,9)} {{(4, 7), (6, 6), (3, 0), (2, 1), (6, 2), (1, 6), (0, 3),
      (8, 5), (7, 2), (1, 2), (7, 4), (6, 7), (7, 6), (5, 8), (5, 0), (2, 2),
      (4, 1), (2, 6), (1, 4), (0, 5), (2, 7), (8, 3), (6, 1), (3, 8)}} {{{(3,
      8)/(3, 7)},{(1, 2)/(1, 3)},{(0, 3)/(1, 3)},{(7, 2)/(7, 3)},{(7, 4)/(8,
      4)},{(3, 0)/(3, 1)},{(2, 7)/(3, 7)},{(7, 6)/(7, 5)},{(2, 2)/(1, 1)},{(5,
      8)/(5, 7)},{(2, 6)/(1, 7)},{(1, 4)/(0, 4)},{(6, 7)/(5, 7)},{(6, 2)/(7,
      1)},{(6, 6)/(7, 7)},{(4, 7)/(4, 8)},{(5, 0)/(5, 1)},{(6, 1)/(5, 1)},{(0,
      5)/(1, 5)},{(4, 1)/(4, 0)},{(1, 6)/(1, 5)},{(8, 5)/(7, 5)},{(8, 3)/(7,
      3)},{(2, 1)/(3, 1)}}}
    }
  \end{subfigure}
\caption{Oscillation}
\end{figure}



Nous avons modifié l'algorithme de beaucoup de manières pour essayer de
résoudre ce problème. Nous essayions d'identifier les cas d'oscillation
(locales et globales), et sommes arrivés à une solution qui semble marcher dans
tous les cas. C'est de nouveau les instantanés (f), (g), (m) et (n) qui posent
problème~: dans les cas (f) et (m), on n'effectue le mouvement que si les
cellules ($i-2$,$j$) et ($i-2$,$j+1$) sont toutes les deux soit vides soit
pleines. Pour les cas (g) et (n), ce sont les cellules ($i+2$,$j$) et
($i+2$,$j+1$). Notez que ces cellules ne sont pas à portée directe des robots
(elles sont à distance 2.) Concernant la mise en \oe{}uvre de cette
modification, nous avons d'abord essayer d'étendre la solution du problème
précédent en ajoutant deux nouvelles rondes~: l'une pour se déplacer à portée
des cellules concernés~; l'autre pour revenir avec l'information. Mais cela est
tout simplement faux car lors de cette première ronde, \emph{tous} les robots
qui se trouvent dans les cas (f), (g), (m) ou (n) vont se déplacer, et
probablement remplir des cellules à vérifier par d'autres robots. Or la
présence ou non des robots dans ces cellules concerne l'état précédent tout
déplacement de robots. La solution proposée ne peut donc se réaliser qu'avec
des robots ayant un rayon de visibilité supérieur ou égal à 2. Nous continuons
tout de même a utiliser les instantanés des positions des voisins à distance 1
(figure \ref{fig:cas_mvt}), mais lorsqu'un robot se trouve dans le cas (f),
(g), (m) ou (n), il regarde en plus les deux positions concernés sans se
soucier des autres robots à distance 2. Même si cela est frustrant de devoir
augmenter les capacités des robots pour si peu, leur visibilité est toujours
limitée, et nous restons cohérent avec les contraintes que nous nous sommes
imposés. Les cas (f), (g), (m) et (n) modifiés sont donc présentés en figure
\ref{fig:fgmn_modif}. \\

\begin{figure}[h!]
  \centering
  \begin{subfigure}[b]{0.20\textwidth}
    \resizebox{\linewidth}{!} {
      \spacee {(4,3)} {{(2,1),(1,0),(3,1)}}             {{{(2,1)/(2,0)}}}
    }
  \end{subfigure}
  \begin{subfigure}[b]{0.20\textwidth}
    \resizebox{\linewidth}{!} {
      \spacee {(4,3)} {{(2,1),(1,0),(3,1),(0,0),(0,1)}}             {{{(2,1)/(2,0)}}}
    }
  \end{subfigure}
  \begin{subfigure}[b]{0.20\textwidth}
    \resizebox{\linewidth}{!} {
      \spacee {(4,3)} {{(1,1),(2,0),(0,1)}}             {{{(1,1)/(1,0)}}}
    }
  \end{subfigure}
  \begin{subfigure}[b]{0.20\textwidth}
    \resizebox{\linewidth}{!} {
      \spacee {(4,3)} {{(1,1),(2,0),(0,1),(3,0),(3,1)}}             {{{(1,1)/(1,0)}}}
    }
  \end{subfigure}
  \begin{subfigure}[b]{0.20\textwidth}
    \resizebox{\linewidth}{!} {
      \spacee {(4,3)} {{(2,1),(1,0),(3,1),(3,2)}}       {{{(2,1)/(2,0)}}}
    }
  \end{subfigure}
  \begin{subfigure}[b]{0.20\textwidth}
    \resizebox{\linewidth}{!} {
      \spacee {(4,3)} {{(2,1),(1,0),(3,1),(3,2),(0,0),(0,1)}}       {{{(2,1)/(2,0)}}}
    }
  \end{subfigure}
  \begin{subfigure}[b]{0.20\textwidth}
    \resizebox{\linewidth}{!} {
      \spacee {(4,3)} {{(1,1),(2,0),(0,1),(0,2)}}       {{{(1,1)/(1,0)}}}
    }
  \end{subfigure}
  \begin{subfigure}[b]{0.20\textwidth}
    \resizebox{\linewidth}{!} {
      \spacee {(4,3)} {{(1,1),(2,0),(0,1),(0,2),(3,0),(3,1)}}       {{{(1,1)/(1,0)}}}
    }
  \end{subfigure}
  \caption{cas (f), (g), (m) et (n) modifiés}
  \label{fig:fgmn_modif}
\end{figure}



Au final, nous avons un algorithme qui fonctionne dans tous les cas testés.
Nous définissons deux fonctions pour simplifier l'écriture formelle de
l'algorithme~:
\begin{itemize}
  \item $prendre\_instantane()$ qui prend un instantané des voisins accessibles
  et effectue la rotation nécessaire pour qu'il soit dans un des cas de
  voisinage de la figure \ref{fig:cas}~;
  \item $rotation180(N_k)$ qui effectue une rotation de $180^\circ{}$ de
  l'instantané passé en paramètre.
\end{itemize}
La figure \ref{fig:algo} donne le pseudo-code de l'algorithme final. \\

\begin{algorithm}[H]
  $finish \leftarrow False$\;
  $k \leftarrow 0$\;
  \While{not $finish$} {
    $N_k \leftarrow get\_neighbors()$\;
    \uIf{$k \% 2 = 0$} {
      % test d'arrêt
      \uIf{$N_k$ doesn't contain any neighbor or \\
          ($N_k$ is case $(a)$ or $(b)$ and $N_{k-2}=rotate180(N_k)$) or \\
          ($N_k$ is case $1.3.2$         and $N_{k-2}=rotate90 (N_k)$) } {
          $finish \leftarrow True$\;
      }
      % mouvement si il n'y a personne en (i,j-2) ou (i+1,j-2)
      \ElseIf{$(i, j-2)$ and $(i+1, j-2)$ are both empty or both full} {
        move according to $N_k$\;
      }
    }
    \Else{
      \If{($N_{k-1}$ is case $(f)$, $(g)$, $(m)$ or $(n)$) and
          ($(i,j+1)$, $(i-1,j)$, $(i-1,j+1)$ are all empty)} {
          move to $(i-1,j)$\;
      }
    }
    $k \leftarrow k+1$\;
  }
  \caption{}
\end{algorithm}


Au cours du développement de l'algorithme, j'ai effectué une étude sur le temps
de convergence des robots lorsqu'ils sont agencés en rectangles ou en blocs
(rectangle rempli.) J'ai extrait des motifs qui se répétaient durant le
processus, me permettant de calculer le temps qu'il fallait pour les robots
pour se rassembler de manière très précise (à la ronde près.) Mais l'algorithme
à été modifié depuis, et les robots ne respectent plus les motifs identifiés
précédemment, rendant l'étude obsolète. Je ne l'exposerai donc pas dans ce
rapport. \\

Une fois qu'un algorithme pour le \GatheringProblem a été identifié, nous nous
sommes attaché à prouver sa correction. C'est à dire, apporter une preuve
mathématique qui montre que le rassemblement s'effectue bien en un nombre fini
de rondes si on utilise cet algorithme. En réalité, ce processus n'est pas si
linéaire. Dès qu'un premier algorithme semblait fonctionner, nous avons
commencé à réfléchir à une preuve. Cela nous amenai à voir que l'algorithme
était faux, nous obligeant à le modifier, et ainsi de suite. L'algorithme de la
figure \ref{fig:algo} est donc la dernière version et a été testé sur de
nombreux cas aléatoires et connu comme posant problème (arrangement en cercles
principalement.) La preuve de cet algorithme est à ce jour bien avancée mais
pas totalement terminée. Si nous succédons à prouver la correction de
l'algorithme, nous pourrons publier un article (en cours de rédaction) exposant
ces résultats. Je vais vous présenter ici l'avancement de la preuve, des
éléments techniques, ainsi que ce qu'il manque pour la terminer. \\

L'idée principale de la preuve est de considérer le plus petit rectangle
englobant tout les robots à un instant $t$, noté $BB(t)$ (\textit{Bounding
Box}), et de montrer que ce rectangle diminue de taille après un nombre fini de
rondes. Cela se réduit à considérer la ligne la plus haute de ce rectangle
(\textit{topmost row}) et de montrer qu'elle \textit{descend} en un nombre fini
de rondes. Pour cela, nous montrons dans un premier temps que s'il n'y a qu'un
seul robot sur cette ligne, celle-ci descend au bout d'un nombre fini d'étapes.
Puis, dans un deuxième temps, on considère le robot le plus à gauche se situant
sur cette ligne (\textit{leftmost robot}) à un instant $t$, et on montre qu'au
bout d'un nombre fini de rondes, le robot le plus à gauche s'est déplacé vers
la droite et ne se déplacera plus sur la gauche. Le robot considéré n'est pas
forcément le même, c'est le robot le plus à gauche sur la ligne la plus haute à
un instant $t$. Analogiquement, on réitère la preuve avec le robot le plus à
droite qui se déplace sur la gauche, et l'ensemble prouve bien que la ligne la
plus haute descend au bout d'un nombre fini de rondes. \\

Soit $r(t)$ le robot (seul) sur la ligne la plus haute de $BB(t)$ à l'étape
$t$. S'il y a plus d'un robot sur cette ligne, $r(t)$ n'existe pas. Les
coordonnées de $r(t)$ sont notés $X(t)$ et $Y(t)$. Nous assumerons en général
que $Y(t)=0$ sauf mention contraire. L'arrangement global des robots à l'étape
$t$ est notée $C(t)$. Si $C(t)$ est un cas terminal de l'algorithme, alors
c'est appelé un arrangement ou une configuration \Gathered (rassemblée.)

\begin{prop}
Si $r(t)$ exixte et est en $(i,0)$, alors il y avait un robot en
$(i-1,0)$, $(i,0)$ ou en $(i+1,0)$ à l'étape $t-1$.
\end{prop}

\begin{prop}
\`A l'étape $t$, si un robot $w$ n'est pas dans le voisinage visible de $r(t)$,
alors $w$ ne pourra pas être dans la ligne la plus haute à l'étape $t+1$.
\end{prop}

Ces deux propositions implique qu'il suffit d'étudier le voisinage de $r(t)$
pour des nouveaux robots sur la ligne la plus haute pour déterminer si celle-ci
est descendu.

\begin{lem}
Si $r(t)$ existe et $C(t)$ n'est pas \Gathered alors il existe une constante
$c$ telle qu'après $c$ rondes, soit $BB(t+c) \subset BB(t)$ soit on est arrivé
dans une configuration \Gathered.
\end{lem}

On prouve d'abord la première partie~: lorsque il n'y a qu'un seul robot sur la
ligne la plus haute.

\begin{proof}
On définit le graphe $G_{single}(V_{single}, E_{single})$ commt suit~:
\begin{itemize}
  \item $V_{single}$ : instantanés de rayon 1 de $r(t)$
  \item $(u,v) \in E_{single}$ si $u$ est le voisinage de $r(t)$ et $v$ est le
  voisinage de $r(t+1)$ tel que $Y(t)=Y(t+1)$ et $C(t+1)$ n'est pas \Gathered
\end{itemize}
Autrement dit, on créé un graphe qui représente les mouvements du robot sur la
ligne la plus haute. Ce graphe a été généré par programmation en énumérant tous
les cas possibles de voisinage du robot après une étape. Pour cela, on énumère
tous les arrangement possibles de l'espace sur un rectangle suffisamment grand
tel que le robot au milieu est soit sur la ligne la plus haute. Puis on lance
l'algorithme sur une étape, et on récupère l'entourage du robot. L'image du
graphe est générée en utilisant le programme \textit{dot} et est présentée en
figure \ref{fig:graph_single_all} en annexe. Les instantanés sur les n\oe{}uds
sont représenté par une matrice où 0 et 1 représentent respectivement l'absence
et la présence d'un robot.

Si un n\oe{}ud n'a pas d'arête sortante, cela signifie que lorsque $r(t)$ est
dans le cas représenté par ce n\oe{}ud, à $t+1$, soit la ligne est descendue,
soit on a atteint une configuration \Gathered. On remarque que le graphe
comporte des cycles~; si un chemin cyclique est suivi par l'algorithme, alors
la ligne la plus haute peut ne jamais descendre. Néanmoins, les arêtes du
graphe ne représentent qu'une seule étape de l'algorithme. Nous allons donc
montrer qu'en réalité l'algorithme ne peut suivre aucun cycles. Il nous suffit
pour cela d'étudier seulement trois chemins, et le lemme sera prouvé. \\

Dans ce rapport, je me contenterai de montrer l'impossibilité d'un seul chemin.
Les autres suivent le même raisonnement, je suis à votre disposition si vous
êtes intéressé par la preuve complète. Toutes les affirmations cités ici ont
été vérifiées par programmation par énumération. \\

Notons de $A$ à $G$ les 7 n\oe{}uds de $G_{single}$ du haut vers le bas puis de
la gauche vers la droite.

Considérons le chemin ($B \rightarrow E \rightarrow D$).
Le seul voisinage possible pour que l'algorithme suive l'arête $B \rightarrow
E$ est le suivant (c'est un déplacement vers la droite)~: \\
\spacee {(5,4)} {{(1,2),(1,1),(2,1),(3,0),(4,0),(4,1)}} {{}} \\
Selon la rêgle (n), le robot en ($X(t)+1,Y(t)+1$) va descendre et occuper la
cellule ($X(t+1),Y(t+1)+2$). Or, pour que l'algorithme suive l'arête $E
\rightarrow D$ à $t+1$, la cellule ($X(t+1),Y(t+1)+2$) doit être vide. Par
conséquent, le chemin ($B \rightarrow E \rightarrow D$) n'existe dans aucune
exécution de l'algorithme. Analogiquement, le chemin $D \rightarrow E
\rightarrow B$ n'existe pas non plus~; ainsi que les chemins ($B \rightarrow E
\rightarrow B$) et ($D \rightarrow E \rightarrow D$) du fait de la symétrie des
n\oe{}uds $B$ et $D$. \\

Nous montrons de manière similaire l'impossibilité des chemins ($B \rightarrow
C \rightarrow D$) et ($B \rightarrow D \rightarrow B$), et par analogie et
symétrie, tous les cycles de $G_{single}$.

\end{proof}

Donc nous avons bien prouvé une première partie du lemme qui implique que s'il
y a qu'un seul robot sur la ligne la plus haute, celle-ci descend après un
nombre fini de rondes.

Nous abordons la deuxième partie de la preuve de manière similaire. Mais elle
en demeure plus compliquée. Notons $g(t)$ le robot le plus à gauche à l'étape
$t$, et redéfinissons $X(t)$ et $Y(t)$ comme coordonnées de $g(t)$.

\begin{prop}
Si $C(t)$ n'est pas \Gathered alors il existe une constante $c$ telle qu'après
$c$ rondes, soit $C(t)$ est \Gathered, soit $g(t+c)$ est plus à droite que
$g(t)$, i.e. $X(g(t+c)) > X(g(t))$.
\end{prop}

\begin{proof}

Nous définissons de manière similaire $G_{leftmost\_mid}$ le graphe des
mouvement de $g(t)$ en ajoutant une contrainte~: les arêtes représentent des
déplacement \textit{sur place}. C'est à dire que le robot au milieu du cas au
n\oe{}ud origine a les mêmes coordonnées que celui au milieu du cas au n\oe{}ud
destination.  Le graphe est présenté en figure \ref{fig:graph_leftmost_mid} en
annexe. Pour prouver le lemme, il faut démontrer l'impossibilité des chemins
cycliques dans ce graphe, puis étudier les cas de déplacement vers la gauche.
\`A ce jour, seul la première partie a été démontrée. Je vais vous présenter
ici comment nous avons traité les cycles n'impliquant qu'un seul n\oe{}ud, avec
celui dessiné en haut à gauche pour exemple. Les autres sont traités comme pour
$G_{single}$. \\

On va regarder la ligne d'en dessous. Pour que ce cycle s'effectue, il ne faut
pas que le robot $(X(t),Y(t))$ ne se déplace. Et comme $Y(t)$ représente la
coordonnée de la ligne la plus haute, il n'y a aucun robot au dessus, donc (dû
au cas (f) modifié) y a obligatoirement un robot en $(X(t)-2,Y(t)+1)$. Par
conséquent, ce dernier n'a pas de voisin à distance 1 au dessus de lui. On va
donc reprendre la preuve avec pour $g(t)$ le robot le plus à gauche de la ligne
d'en dessous. Ce processus a forcément une fin, car le nouveau robot considéré
est plus à gauche que le précédent, donc à un moment donné, la boucle ne sera
pas effectuée car intuitivement, on arrivera sur le bord gauche de la figure.
Donc si on démontre la proposition avec $G_{leftmost\_mid}$ privé de la boucle
considérée ici, elle sera vrai pour $G_{leftmost\_mid}$ complet, et donc le
lemme sera également vrai.

\end{proof}

Le graphe représentant les mouvement à gauche de $g(t)$ est aussi présenté en
annexe sur la figure \ref{fig:graph_leftmost_left}. Celui des mouvement de
droite est très complexe et ne présente pas grand intérêt pour ce rapport, il
sera donc omis.

Un dernier point concerne la complexité de l'algorithme. Si on arrive à compter
le nombre de rondes nécessaire pour la ligne la plus haute à descendre, on aura
la complexité totale du rassemblement assez facilement. Notre intuition nous
dirige vers une complexité linéaire en la taille des robots (il y en a au plus
$l*h$ si $l$ et $h$ sont les dimensions de $BB(0)$.) \\

Cela conclue donc la partie recherche théorique du TER. Bien que l'unité
d'enseignement ne sois terminée, je continue à prouver la correction de
l'algorithme en partenariat avec mon tuteur. Je vais maintenant vous parler de
l'aspect programmation du projet.

\subsection{Programmation}

J'ai donc utilisé le langage Python et la bibliothèque TkInter, en me basant
sur la charte de codage conseillée par Python~: PEP8~; le tout en anglais,
permettant une plus grande maintenabilité. L'objectif était de créer une
interface graphique permettant de visualiser les algorithmes distribués
développés sur papier. \\

L'interface graphique nécessaire est très sommaire~: un canevas contenant une
grille fixe, et deux boutons~:
\begin{itemize}
  \item \emph{next} pour avancer d'une étape dans l'algorithme
  \item \emph{prev} pour reculer
\end{itemize}
Les clics de la souris sur le canevas permettent d'ajouter des robots à la
volée sur l'espace où d'en enlever. Un autre bouton \emph{clear} a été ajouté
pour supprimer tous les robots. Avec ceci, il est très facile d'effectuer des
tests sur des exemples d'arrangements de robots qui pourrait poser problème. La
figure \ref{fig:screenshot} en annexe montre une capture d'écran de
l'application avec plusieurs arrangements de robots de tests en cours de
rassemblement. Les robots colorés en rouge sont ceux qui se sont détectés comme
étant dans un état quescient. \\

L'ensemble des robots est stocké en mémoire dans un conteneur Python appelé
\texttt{set}. Il sert à stocker des ensembles, c'est à dire une liste d'objet
uniques non ordonnée. Cela est suffisant car on a dit que les robots sur une
même position peuvent être vus comme un seul et même robot.  Pour éviter de
calculer plusieurs fois une étape (par exemple lors de clics successifs sur
\emph{next} et \emph{prev}), on stocke une liste d'ensemble de robots indexés
par le numéro de la ronde. C'est la classe \texttt{Space} qui contient cette
liste, et elle fourni les méthodes d'avancement des robots ainsi que des
méthodes pour le remplissage de l'espace.

Des configurations de robots peuvent être écrites dans un fichier texte sous
forme binaire, qui sera lu et automatiquement stocké en mémoire par le
programme. Il suffit alors de choisir quels arrangements utiliser dans l'espace
avant de lancer le programme. Il a donc fallu écrire une fonction qui traduit
les coordonnées binaires des robots (décalages) en coordonnées cartésiennes (x
et y). Voici l'exemple de la figure \ref{fig:oscillation}~: \\

\begin{verbatim}
spiky-square
0 0 0 0 1 0 0 0 0
0 1 1 1 0 1 1 1 0
0 1 0 0 0 0 0 1 0
0 1 0 0 0 0 0 1 0
1 0 0 0 0 0 0 0 1
0 1 0 0 0 0 0 1 0
0 1 0 0 0 0 0 1 0
0 1 1 1 0 1 1 1 0
0 0 0 0 1 0 0 0 0
\end{verbatim}

En effet, tout les calculs sur les arrangements de robots sont en binaire. Ce
n'est que lorsqu'on entre en contact avec l'interface graphique que l'espace
est converti. Il a notamment fallu écrire une fonction de génération de
configurations aléatoires. L'idée est de commencer par un espace empli de
robots (une suite de 1), puis de les enlever (mettre un bit à 0) au fur et à
mesure jusqu'à ce qu'on en ai le nombre désiré. Et à chaque fois qu'on met un
bit à 0, on test si l'espace est encore connexe, si oui on continue, sinon, on
remet bit à 1. Pour tester la connexité, on utilise un parcours en largeur des
robots avec une pile en commençant par un le bit à 1 de poids le plus faible.
\\

Concernant l'implémentation de l'algorithme, tout d'abord, les cas de mouvement
sont stockés dans un fichier à part comme dans la figure \ref{fig:cas_mvt}. Les
rotations sont effectués automatiquement et le tout est stocké dans un
dictionnaire Python permettant d'avoir directement le mouvement à effectuer à
partir du voisinage. Le format est le suivant~:
Les trois premières lignes correspondent à la version binaire de l'instantané,
et la quatrième contient les coordonnées relatives au mouvement à effectuer.
Les instantanés sont séparés par des lignes vides. Voici l'exemple du cas (d)~:
\begin{verbatim}
0 0 0
0 1 0
1 0 1
1, 0
\end{verbatim}
Cela permet une grande flexibilité par rapport aux instantanés utilisés pour
l'algorithme.

Une première différence avec l'algorithme théorique est dû à la parallélisation
des calculs effectués par les robots. Notre programme est séquentiel, il faut
donc parcourir tous les robots deux fois au moins~: une pour prévoir la
prochaine position des robots, une deuxième pour les déplacer.  Lorsqu'un clic
est effectué sur le bouton \emph{next}, on récupère l'entourage de chaque robot
sous forme binaire ainsi que leur mouvement associé. Puis on change les
coordonnées des robots en gardant à part ceux qui étaient dans les cas (f),
(g), (m) et (n). Enfin, pour ces derniers, on test s'ils sont déconnectés avec
une opération binaire entre un masque et l'entourage récupéré, et on re-change
leur coordonnées si nécessaire. \\

Nous avons beaucoup utilisé la programmation pour la preuve de l'algorithme,
notamment pour générer les graphes et vérifier les contraintes d'entourages
pour traverser certaines arêtes (voir preuve.) Il a fallu écrire une fonction
assez complexe manipulant des bits~: à partir d'un champs de bit
interprété comme une matrice de 0 et de 1, récupérer une sous matrice dont les
dimensions sont données. C'est peut-être la fonction qui m'a donné le plus de
mal mais qui est très courte et fonctionne au final très bien. \\

Le fait que nous ne manipulions que des champs de bits rend la génération de
configurations très facile~: il suffit de compter en décimal, puis de convertir
en binaire. Néanmoins, pour générer les graphes, il faut générer tous les
arrangements sur un rectangle $7*7$, et à chaque fois, tester la connexité,
avancer d'une étape, puis récupérer le nouveau voisinage~; et ce pour chaque
instantané sujet à mouvement. Pour ce travail, il a fallu optimiser un maximum
toutes ces opérations, en changeant les structures de données, en minimisant
les opérations de vérification (de terminaison par exemple), etc... La
génération de $G_{leftmost\_mid}$ a durée environ 45 minutes. C'est pour cela
que même si ça nous aurai aidé pour la preuve de correction, nous n'avons pas
généré les graphes pour deux étapes de l'algorithme. Ça aurait multiplié le
temps de calcul de façon exponentiel, car il aurait fallu générer les
arrangements sur des plus gros rectangles. \\

Pour terminer, j'ai ajouter temporairement un système de \emph{dump} qui
affiche sur la sortie standard les positions des robots présents sur l'espace,
et si besoin les prochains mouvements. Le tout formaté pour \emph{tikz}. Je
pouvais ainsi très facilement dessiner sur le rapport des configurations de
robots venant directement du programme.
