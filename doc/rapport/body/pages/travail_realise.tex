\section{Travail réalisé}

Cette section retrace le travail que j'ai réalisé durant ce projet, les
difficultés que j'ai rencontré ainsi que les solutions apportées.

\subsection{Recherche fondamentale}

Le processus de recherche a nécessité plusieurs étapes, et est encore en cours.
Dans un premier temps, il a fallu déterminer les cas de voisinage (instantanés)
dans lesquels les robots effectuent un mouvement. On considère d'abord que les
robots n'ont qu'une visibilité de 1, c'est à dire qu'ils ne voient que les
robots qui sont dans des cellules adjacentes. La figure \ref{fig:cas} en annexe
montre tous les cas de voisinage à distance 1 qu'un robot peut rencontrer. Il
n'est pas nécessaire de traiter les cas analogues correspondant à des rotations
de $90^\circ{}$, $180^\circ{}$ et $270^\circ{}$ de chacun de ces cas.  En
effet, lorsqu'un robot reçoit un instantané qui ne fait pas parti des cas
exposés, il effectue la rotation nécessaire, et applique l'algorithme selon le
cas ayant subi la rotation. \\

L'idée de base pour le rassemblement est d'\textit{arrondir les angles}. La
figure \ref{fig:cas_mvt} montre les cas sujet à mouvements.

\begin{figure}
  \centering
  % \captionsetup[subfigure]{labelformat=empty}
  \begin{subfigure}[b]{0.15\textwidth}
    \resizebox{\linewidth}{!} {
      \spacee {(3,3)} {{(1,1),(2,0)}}                   {{{(1,1)/(2,0)}}}
    }
    \caption{}
  \end{subfigure}
  \begin{subfigure}[b]{0.15\textwidth}
    \resizebox{\linewidth}{!} {
      \spacee {(3,3)} {{(1,1),(1,0)}}                   {{{(1,1)/(1,0)}}}
    }
    \caption{}
  \end{subfigure}
  \begin{subfigure}[b]{0.15\textwidth}
    \resizebox{\linewidth}{!} {
      \spacee {(3,3)} {{(1,1),(2,0),(1,0)}}             {{{(1,1)/(1,0)}}}
    }
    \caption{}
  \end{subfigure}
  \begin{subfigure}[b]{0.15\textwidth}
    \resizebox{\linewidth}{!} {
      \spacee {(3,3)} {{(1,1),(0,0),(2,0)}}             {{{(1,1)/(1,0)}}}
    }
    \caption{}
  \end{subfigure}
  \begin{subfigure}[b]{0.15\textwidth}
    \resizebox{\linewidth}{!} {
      \spacee {(3,3)} {{(1,1),(0,0),(1,0)}}             {{{(1,1)/(1,0)}}}
    }
    \caption{}
  \end{subfigure}
  \begin{subfigure}[b]{0.15\textwidth}
    \resizebox{\linewidth}{!} {
      \spacee {(3,3)} {{(1,1),(0,0),(2,1)}}             {{{(1,1)/(1,0)}}}
    }
    \caption{}
  \end{subfigure}
  \begin{subfigure}[b]{0.15\textwidth}
    \resizebox{\linewidth}{!} {
      \spacee {(3,3)} {{(1,1),(2,0),(0,1)}}             {{{(1,1)/(1,0)}}}
    }
    \caption{}
  \end{subfigure}
  \begin{subfigure}[b]{0.15\textwidth}
    \resizebox{\linewidth}{!} {
      \spacee {(3,3)} {{(1,1),(0,1),(1,0)}}             {{{(1,1)/(0,0)}}}
    }
    \caption{}
  \end{subfigure}
  \begin{subfigure}[b]{0.15\textwidth}
    \resizebox{\linewidth}{!} {
      \spacee {(3,3)} {{(1,1),(0,0),(1,0),(2,0)}}       {{{(1,1)/(1,0)}}}
    }
    \caption{}
  \end{subfigure}
  \begin{subfigure}[b]{0.15\textwidth}
    \resizebox{\linewidth}{!} {
      \spacee {(3,3)} {{(1,1),(2,0),(1,0),(2,1)}}       {{{(1,1)/(1,0)}}}
    }
    \caption{}
  \end{subfigure}
  \begin{subfigure}[b]{0.15\textwidth}
    \resizebox{\linewidth}{!} {
      \spacee {(3,3)} {{(1,1),(0,0),(1,0),(2,1)}}       {{{(1,1)/(1,0)}}}
    }
    \caption{}
  \end{subfigure}
  \begin{subfigure}[b]{0.15\textwidth}
    \resizebox{\linewidth}{!} {
      \spacee {(3,3)} {{(1,1),(2,0),(1,0),(0,1)}}       {{{(1,1)/(1,0)}}}
    }
    \caption{}
  \end{subfigure}
  \begin{subfigure}[b]{0.15\textwidth}
    \resizebox{\linewidth}{!} {
      \spacee {(3,3)} {{(1,1),(0,0),(2,1),(2,2)}}       {{{(1,1)/(1,0)}}}
    }
    \caption{}
  \end{subfigure}
  \begin{subfigure}[b]{0.15\textwidth}
    \resizebox{\linewidth}{!} {
      \spacee {(3,3)} {{(1,1),(2,0),(0,1),(0,2)}}       {{{(1,1)/(1,0)}}}
    }
    \caption{}
  \end{subfigure}
  \begin{subfigure}[b]{0.15\textwidth}
    \resizebox{\linewidth}{!} {
      \spacee {(3,3)} {{(1,1),(0,0),(2,1),(2,0)}}       {{{(1,1)/(1,0)}}}
    }
    \caption{}
  \end{subfigure}
  \begin{subfigure}[b]{0.15\textwidth}
    \resizebox{\linewidth}{!} {
      \spacee {(3,3)} {{(1,1),(0,0),(0,1),(2,0)}}       {{{(1,1)/(1,0)}}}
    }
    \caption{}
  \end{subfigure}
  \begin{subfigure}[b]{0.15\textwidth}
    \resizebox{\linewidth}{!} {
      \spacee {(3,3)} {{(1,1),(0,0),(1,0),(2,0),(2,1)}} {{{(1,1)/(1,0)}}}
    }
    \caption{}
  \end{subfigure}
  \begin{subfigure}[b]{0.15\textwidth}
    \resizebox{\linewidth}{!} {
      \spacee {(3,3)} {{(1,1),(0,0),(1,0),(2,0),(0,1)}} {{{(1,1)/(1,0)}}}
    }
    \caption{}
  \end{subfigure}
\caption{Cas de voisinnage sujet à mouvement}
\end{figure}



Ces cas semblent assez naturels, cependant ce n'est pas aussi simple. Par
exemple, examinons de plus près les cas (f) et (g). Si les robots se déplacent
dans ces cas, il existe une infinité d'arrangements des robots qui mènent à un
graphe non connexe. Or, si ils ne se déplacent pas, il existe une infinité
d'arrangements quescient alors que les robots ne sont pas du tout rassemblés.
Il en est de même pour les cas (m) et (n). Les figures \ref{fig:deconnexion} et
\ref{fig:quescient} montrent des exemples de cas symétriques où les robots sont
respectivement déconnectés et quescient si les cas (f) et (g) ne sont
respectivement utilisés et non utilisés. Même s'il n'était pas nécessaire de le
prouver, cela montre que le \GatheringProblem ne pourra se résoudre avec des
robots sans mémoire (\textit{oblivious}).

\begin{figure}
  \centering
  \begin{subfigure}[b]{0.40\textwidth}
    \resizebox{\linewidth}{!} {
      \spacee {(9,9)} {{(7, 3),(1, 3),(6, 6),(3, 0),(6, 2),(5, 1),(3, 7),(0,
      3),(8, 5),(5, 8),(4, 0),(4, 8),(1, 5),(5, 0),(2, 2),(5, 7),(2, 6),(0,
      4),(7, 5),(0, 5),(8, 3),(3, 1),(3, 8),(8, 4)}} {{{(0, 3)/(1, 4)},{(3,
      1)/(2, 1)},{(5, 7)/(6, 7)},{(3, 0)/(4, 1)},{(3, 8)/(4, 7)},{(0, 5)/(1,
      4)},{(8, 3)/(7, 4)},{(1, 3)/(1, 2)},{(3, 7)/(2, 7)},{(5, 1)/(6, 1)},{(5,
      8)/(4, 7)},{(8, 5)/(7, 4)},{(7, 5)/(7, 6)},{(5, 0)/(4, 1)},{(7, 3)/(7,
      2)},{(1, 5)/(1, 6)}}}
    }
  \end{subfigure}
  \begin{subfigure}[b]{0.40\textwidth}
    \resizebox{\linewidth}{!} {
      \spacee {(9,9)} {{(1, 2),(2, 7),(4, 7),(2, 6),(6, 7),(6, 6),(6, 1),(4,
      8),(1, 4),(7, 4),(6, 2),(1, 6),(2, 1),(2, 2),(0, 4),(8, 4),(4, 1),(7,
      6),(7, 2),(4, 0)}} {{}}
    }
  \end{subfigure}
  \caption{Déconnexion}
\end{figure}


\begin{figure}
  \centering
  \resizebox{0.30\textwidth}{!} {
    \spacee {(7,7)}{{(6, 4),(2, 6),(4, 6),(2, 0),(3, 0),(0, 2),(6, 3),(1, 5),(6,
    2),(3, 6),(0, 4),(5, 1),(0, 3),(4, 0),(1, 1),(5, 5)}} {{}}
  }
\caption{Quescient}
\end{figure}



Pour résoudre ce problème, il faut réussir à détecter la déconnexion, puis
utiliser la mémoire des robots pour revenir à la position précédente. Cela
nécessite deux rondes~:
\begin{enumerate}[1]
  \item Regarder son entourage et se déplacer en utilisant les cas (f), (g),
  (m) ou (n) s'il le faut, et enregistrer le mouvement.
  \item Détecter la déconnexion si on était dans un des cas (f), (g), (m) ou
  (n) et se déplacer à la position précédente si nécessaire.
\end{enumerate}
Il faut que tous les robots exécutent ces deux rondes, et que les robots ne se
déplacent pas à la deuxième ronde s'ils ne sont pas déconnectés. Cela peut se
réaliser facilement avec un compteur de rondes. Comme les robots commencent
tous en même temps, on peut définir que les rondes paires correspondent à la
première étape, et les rondes impaires à la deuxième. Ainsi, cela double le
nombre de rondes sans affecter la complexité de l'algorithme.

Si un robot en position ($i$,$j$) est dans le cas (f) ou (m) à la première
étape, la déconnexion est détectée dans la deuxième étape par l'absence totale
de robots aux positions ($i+1$,$j$), ($i+1$,$j-1$) et ($i$,$j-1$). Pour les cas
(g) et (n) ce sont les positions ($i-1$,$j$), ($i-1$,$j-1$) et ($i$,$j-1$). \\

Ceci étant réglé, un autre problème se pose~: il existe des paires
d'arrangements circulaires de robots qui alternent entre eux. Autrement dit,
les robots ne sont pas quescient mais ne se rassemblent jamais. La figure
\ref{fig:oscillation} donne un tel exemple d'oscillation.

\begin{figure}
  \centering
  \begin{subfigure}[b]{0.40\textwidth}
    \resizebox{\linewidth}{!} {
      \spacee {(9,9)} {{(7, 3), (5, 7), (4, 8), (7, 7), (2, 1), (1, 3), (1, 6),
      (5, 1), (3, 7), (7, 2), (4, 0), (1, 2), (6, 7), (7, 6), (1, 5), (0, 4),
      (1, 1), (7, 1), (7, 5), (2, 7), (6, 1), (3, 1), (1, 7), (8, 4)}} {{{(1,
      5)/(0, 5)},{(5, 7)/(5, 8)},{(5, 1)/(5, 0)},{(0, 4)/(1, 4)},{(7, 7)/(6,
      6)},{(7, 1)/(6, 2)},{(3, 1)/(3, 0)},{(1, 1)/(2, 2)},{(1, 3)/(0, 3)},{(1,
      7)/(2, 6)},{(8, 4)/(7, 4)},{(7, 5)/(8, 5)},{(4, 8)/(4, 7)},{(7, 3)/(8,
      3)},{(3, 7)/(3, 8)},{(4, 0)/(4, 1)}}}
    }
  \end{subfigure}
  \begin{subfigure}[b]{0.40\textwidth}
    \resizebox{\linewidth}{!} {
      \spacee {(9,9)} {{(4, 7), (6, 6), (3, 0), (2, 1), (6, 2), (1, 6), (0, 3),
      (8, 5), (7, 2), (1, 2), (7, 4), (6, 7), (7, 6), (5, 8), (5, 0), (2, 2),
      (4, 1), (2, 6), (1, 4), (0, 5), (2, 7), (8, 3), (6, 1), (3, 8)}} {{{(3,
      8)/(3, 7)},{(1, 2)/(1, 3)},{(0, 3)/(1, 3)},{(7, 2)/(7, 3)},{(7, 4)/(8,
      4)},{(3, 0)/(3, 1)},{(2, 7)/(3, 7)},{(7, 6)/(7, 5)},{(2, 2)/(1, 1)},{(5,
      8)/(5, 7)},{(2, 6)/(1, 7)},{(1, 4)/(0, 4)},{(6, 7)/(5, 7)},{(6, 2)/(7,
      1)},{(6, 6)/(7, 7)},{(4, 7)/(4, 8)},{(5, 0)/(5, 1)},{(6, 1)/(5, 1)},{(0,
      5)/(1, 5)},{(4, 1)/(4, 0)},{(1, 6)/(1, 5)},{(8, 5)/(7, 5)},{(8, 3)/(7,
      3)},{(2, 1)/(3, 1)}}}
    }
  \end{subfigure}
\caption{Oscillation}
\end{figure}



Nous avons modifié l'algorithme de beaucoup de manières pour essayer de
résoudre ce problème. Nous essayions d'identifier les cas d'oscillation
(locales et globales), et sommes arrivés à une solution qui semble marcher dans
tous les cas. C'est de nouveau les instantanés (f), (g), (m) et (n) qui posent
problème~: dans les cas (f) et (m), on n'effectue le mouvement que si les
cellules ($i-2$,$j$) et ($i-2$,$j+1$) sont toutes les deux soit vides soit
pleines. Pour les cas (g) et (n), ce sont les cellules ($i+2$,$j$) et
($i+2$,$j+1$). Notez que ces cellules ne sont pas à portée directe des robots
(elles sont à distance 2). Concernant la mise en \oe{}uvre de cette
modification, nous avons d'abord essayer d'étendre la solution du problème
précédent en ajoutant deux nouvelles rondes~: l'une pour se déplacer à portée
des cellules concernés~; l'autre pour revenir avec l'information. Mais cela est
tout simplement faux car lors de cette première ronde, \textit{tous} les robots
qui se trouvent dans les cas (f), (g), (m) ou (n) vont se déplacer, et
probablement remplir des cellules à vérifier par par d'autres robots. Or la
présence ou non des robots dans ces cellules concerne l'état précédent tout
déplacement de robot. La solution proposée ne peut donc se réaliser qu'avec des
robots ayant un rayon de visibilité supérieur ou égal à 2. Nous continuons tout
de même a utiliser les instantanés des positions des voisins à distance 1
(figure \ref{fig:cas_mvt}), mais lorsqu'un robot se trouve dans le cas (f),
(g), (m) ou (n), il regarde en plus les deux positions concernés sans se
soucier des autres robots à distance 2. Même si cela est frustrant de devoir
augmenter les capacités des robots pour si peu, leur visibilité est toujours
limitée, et nous restons cohérent avec les contraintes que nous nous sommes
imposés. \\

Au final, nous avons un algorithme qui fonctionne dans tous les cas testés.
Nous définissons deux fonctions pour simplifier l'écriture formelle de
l'algorithme~:
\begin{itemize}
  \item $prendre\_instantane()$ qui prend un instantané des voisins accessibles
  et effectue la rotation nécessaire pour qu'il soit dans un des cas de
  voisinage de la figure \ref{fig:cas}~;
  \item $rotation180(N_k)$ qui effectue une rotation de $180^\circ{}$ de
  l'instantané passé en paramètre.
\end{itemize}
La figure \ref{fig:algo} donne le pseudo-code de l'algorithme final. \\

\begin{algorithm}[H]
  $finish \leftarrow False$\;
  $k \leftarrow 0$\;
  \While{not $finish$} {
    $N_k \leftarrow get\_neighbors()$\;
    \uIf{$k \% 2 = 0$} {
      % test d'arrêt
      \uIf{$N_k$ doesn't contain any neighbor or \\
          ($N_k$ is case $(a)$ or $(b)$ and $N_{k-2}=rotate180(N_k)$) or \\
          ($N_k$ is case $1.3.2$         and $N_{k-2}=rotate90 (N_k)$) } {
          $finish \leftarrow True$\;
      }
      % mouvement si il n'y a personne en (i,j-2) ou (i+1,j-2)
      \ElseIf{$(i, j-2)$ and $(i+1, j-2)$ are both empty or both full} {
        move according to $N_k$\;
      }
    }
    \Else{
      \If{($N_{k-1}$ is case $(f)$, $(g)$, $(m)$ or $(n)$) and
          ($(i,j+1)$, $(i-1,j)$, $(i-1,j+1)$ are all empty)} {
          move to $(i-1,j)$\;
      }
    }
    $k \leftarrow k+1$\;
  }
  \caption{}
\end{algorithm}


TODO: proof + étude de rassemblement des carrés ?

\subsection{Programmation}

Dans un algorithme séquentiel,
pour l'implémentation, il faut 2 parcours de tous les robots : prévoir la
prochaine position des robots, puis les déplacer. \\

1 screenshot de l'appli

\subsection{Programmation}
IG
pep8
