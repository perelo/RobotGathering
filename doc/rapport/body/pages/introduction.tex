\section*{Introduction}
\addcontentsline{toc}{section}{Introduction}
Dans le cadre de ma première année de Master informatique à Aix-Marseille
Univsersité (site de Luminy), je suis amené à faire un projet (TER) encadré par
un chercheur d'AMU ou un stage en entreprise. Ayant plus une vocation liée à la
recherche, je me suis orienté vers un projet mêlant programmation et théorie
fondamentale. \\

Le projet rentre dans le domaine de l'algorithmique distribuée~; branche de
l'informatique théorique dont le but est de développer des algorithmes pour des
systèmes distribués impliquant plusieurs processus interconnectés par un
réseau. Ces processus indépendants interagissent et coopèrent entre eux en vue
de réaliser une tâche donnée.  L'idée principal du calcul distribué est que les
processus communiquent entre eux par l'envoi de messages. Néanmoins, nous nous
intéressons ici à un paradigme légèrement différent : les \textit{agents
mobiles}. Ce sont des programmes qui peuvent se déplacer de n\oe{}uds en
n\oe{}uds à l'intérieur du réseau de manière autonome. Bien qu'ils peuvent être
implémentés via le modèle précédent (et font donc parti du calcul distribué),
ils fournissent une abstraction plutôt naturelle pour le développement
d'algorithmes tels que la détection d'intrus, l'exploration d'un réseau
inconnu, ou encore la formation d'un motif par des robots (\textit{robot
pattern formation problem}). Ce dernier problème consiste à arranger les robots
dans le plan pour qu'ils forment et conservent un motif donné. \\

L'objectif ici est de développer et d'implémenter une solution pour un cas
particulier de ce problème~: le rassemblement d'agents mobiles. Aussi appelé le
\GatheringProblem, beaucoup de contributions y ont déjà été apporté, impliquant
souvent un plan continu ou une visibilité infinie. Nous nous restreignons ici à
un espace discret et une visibilité constante des robots. De plus, le système
est synchrone, c'est à dire qu'il existe une horloge globale partagée par tous
les processus. Autrement dit, les algorithmes fonctionneront par
\textit{rondes} (ou \textit{étapes}.) Pour notre cas, cela signifie que tous
les robots décident et se déplacent en même temps.

Le travail à réaliser dans ce projet est donc la recherche des cas de voisinage
et de déplacement des robots pour le \GatheringProblem dans notre contexte. Il
faut également développer une application permettant la visualisation des
robots et de l'algorithme distribué. Cette partie sera faite en utilisant le
langage Python et la librairie TKinter pour l'interface graphique. \\

Dans un premier temps je vais vous présenter plus en détail le sujet et les
outils utilisés~; puis je vous exposerais le travail que j'ai réalisé
développant les différentes étapes de recherche d'un algorithme correct ainsi
que les techniques d'implémentation utilisés. Enfin, je conclurai par un bilan
personnel et professionnel.
