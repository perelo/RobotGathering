
\documentclass[11pt, a4paper]{article}

\usepackage[english,francais]{babel}
\usepackage[utf8]{inputenc}
\usepackage[T1]{fontenc}
\usepackage[pdftex]{graphicx}
\usepackage{setspace}
\usepackage[french]{varioref}
\usepackage{amsfonts}
\usepackage{amssymb}
\usepackage{geometry}
\usepackage{amsthm}
\usepackage[french, ruled]{algorithm2e}
\geometry{margin=2cm}

\title{Calcul du temps de rassemblement dans le cas d'un bloc (draft)}

\begin{document}

\maketitle

\section{Cas paire}

Bloc de taille $n$ par $n$ avec $n \geq 3$.

On ne considère que un côté (les trois autres sont similaires)

$R$ : le plus petit rectangle englobant tous les robots

\paragraph{Première étape} De bloc à disque

Les deux cases prises par les robots sur les bords du côtés se libèrent à
chaque ronde. Donc à chaque ronde, le nombre de robots sur le côté diminue de
2. Le temps pour qu'il ne reste plus que 4 robots est
\[ \frac{n-4}{2} \]

\paragraph{Deuxième étape} Effondrement du disque (ou du cercle)

Le côté ne contient plus que 4 robots adjacents. Donc au bout de deux
étapes, il s'effondre. On se rend compte que tant que $n \geq 3$,
après chaque effondrement, le côté contient de nouveau 4 robots adjacents. Donc,
pour que l'espace devienne un bloc de taille 2 par 2 (cas terminal), il faut
que chaque côté "descendent" de $n/2-1$, soit un nombre d'étapes de
\[ ( \frac{n}{2} -1 ) *2 \]
Puis il faut une dernière ronde pour que les robots se rendent compte qu'ils
ont terminés.

\paragraph{Totaux} Le temps total $T$ pour que un bloc se rassemble
est donc :

\[ T = ( \frac{n}{2} -1 ) *2 + \frac{n-4}{2} + 1 \]
\[ T = n + \frac{n}{2} - 3 \]


\section{Cas impaire}

Le cas impaire est assez similaire :
On utilise des côtés qui ont 3 robots adjacents, et "descendent" en deux
étapes.

\paragraph{Première étape}
\[ \frac{n-3}{2} \]

\paragraph{Deuxième étape}
\[ ( \frac{n-1}{2} ) *2 \]
Pas de ronde en plus après la deuxième étape.

\section{Cas général (paire et impaire)}
\[ n + \left\lceil \frac{n}{2} \right\rceil -3 \]
Formule vérifiée pour des blocs de côtés de taille 3 à 50 de côtés de taille 3
à 100.

\section{Généralisation aux rectangles}

On considère des rectangles de taille $m$ par $n$ avec $m \leq n$.

\[
    \left\lceil \frac{n-4}{2} \right\rceil
  + 2*\left\lfloor \frac{m}{2} \right\rfloor
  -2\ si\ m\ pair
  +1\ si\ m\ ou\ n\ pair
\]

\section{Généralisation aux carrés (blocs vides)}

Tout ce qui a été dis sur les blocs peut s'appliquer (et a été vérifié) aux
carrés et aux rectangles. On peut associer les robots qui forment le contour du
bloc aux robots qui forment le cercle. Puis, on se rend compte que à chaque
étape, les robots correspondant deux à deux prendront la même décision.

\end{document}
