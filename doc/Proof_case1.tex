\newcommand{\Gathered}{\ensuremath{\textsc{gathered}}\xspace}

\section{A single robot on the topmost row}

We denote by $r(t)$ the single robot in the topmost row of the bounding box at step
$t$. If there are more than one robot in the topmost row, $r(t)$ is not defined.
The row and column of the cell occupied by $r(t)$ would be denoted $Y(t)$ and $X(t)$ respectively. In general we will assume $Y(t)=0$ unless otherwise stated. 
The global configuration of robots at time $t$ would be denoted by $C(t)$. If $C(t)$ satisfies the terminating conditions of the algorithm then it is called a \Gathered configuration.

\begin{prop}
If $r(t)$ exists and is on $(0,i)$, then there was a robot on cell $(0,i-1)$, $(0,i)$
or $(0,i+1)$ (or on cell $(-1,i-1)$, $(-1,i)$ or $(-1,i+1)$) at step $t-1$.
\end{prop}

\begin{prop}
At step $t$, if a robot $w$ was not in the neighborhood of $r(t)$, then robot $w$ cannot be on the topmost row in step $(t+1)$.
\end{prop}

The above properties imply that studying the neighborhood of $r(t)$ for new robots on the
topmost row is sufficient to determine if the topmost row moved down.

\begin{lem}
If $r(t)$ exists and the current configuration $C(t)$ is not  \Gathered then there exists a constant $c$ such that 
after $c$ steps, either $BB(t+c) \subset BB(t)$ (i.e. the topmost row moves
down) or a \Gathered configuration is reached.
\end{lem}

\begin{proof}

We define the graph $G_{single}(V_{single}, E_{single})$ as follows :
\begin{itemize}
  \item $V_{single}$ : neighborhood cases of $r(t)$
  \item $(u,v) \in E_{single}$ if $u$ is the neighborhood of $r(t)$ and $v$ is the neighborhood of $r(t+1)$ such that $Y(t)=Y(t+1)$ and the configuration $C(t+1)$ is not \Gathered.
\end{itemize}
The graph generated by considering all possible single step transformations is shown on figure \ref{graph:single}. If a node has no outgoing edges then in the next step $(t+1)$ either there are no robots on row $Y(t)$ or a \Gathered configuration is reached. So the sink nodes in the graph satisfy the lemma. 
We notice multiple cycles in the graph; if any cyclic path is followed by the algorithm then the topmost row might never move down. However, edges of the graph only represent single step transformation. We will show that the cyclic paths are not followed by the algorithm. We will study 4 such paths in the graph and this would be sufficient to prove that the lemma holds. \\
%i.e. $BB(t+c) \subset BB(t)$. \\

\noindent
We denote by $A$ to $G$ the 7 nodes of $G_{single}$ from top to bottom and left to right. \\
A transformation corresponding to an edge in $G_{single}$ is called a \emph{Left-move}, \emph{Mid-move}, or \emph{Right-move}, if  $X(t) > X(t+1)$, $X(t) = X(t+1)$, or $X(t) < X(t+1)$ respectively.\\

%The middle robot in the start node is at $(0,0)$ at $t$.
\begin{enumerate}
\item
Consider the path ($D \rightarrow A \rightarrow B$). 
%Notice that $D \rightarrow A$ and $A \rightarrow B$ are respectively a left and a right move.
Let $t$ be the step at which the algorithm reaches node $D$, i.e. the neighborhood of $r(t)$ is represented by $D$. According to the rule (1.2.1), the robot $r(t)$ would move down. So if the algorithm follows edge $D \rightarrow A$, then some other robot must move up to row $Y(t)$. The only possibility is the robot at ($Y(t)+1,X(t)$) to move up to ($Y(t),X(t)-1$) using rule (1.2.6). This implies that in step $t$, the row $Y(t)+2$ was empty at columns $X(t)-1$ to $X(t)+1$. Thus, in step $(t+1)$, row $Y(t+1)$ is empty at $X(t+1)+1$ and $X(t+1)+2$. So if node $A$ is not a \Gathered configuration, then there must be a robot neighboring the right bottom corner of node $A$. The only possibility is the cell $Y(t+1)+2, X(t+1)$ to be occupied in step $t+1$. Thus in the next step, the topmost row will move down. So, the path ($D \rightarrow A \rightarrow B$) cannot exist in any execution of the algorithm.

\item Consider the path ($D \rightarrow B \rightarrow C$)....

\item Consider the path ($D \rightarrow B \rightarrow E$)....

\item Consider the path ($D \rightarrow B \rightarrow D$)....

\item Consider the path ($E \rightarrow D \rightarrow E$)....

\end{enumerate}
Due to the above arguments, all cyclic paths can be removed from the graph $G_{single}$. This directly implies that the statement of the lemma holds.
\end{proof}

The results of this section show that if there is only robot in topmost (bottom-most) row or equivalently if there is only one robot in leftmost (or rightmost)  column, then the bounding box shrinks within a constant number of steps.